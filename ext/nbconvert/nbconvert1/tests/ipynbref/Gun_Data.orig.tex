%% This file was auto-generated by IPython.
%% Conversion from the original notebook file:
%% tests/ipynbref/Gun_Data.orig.ipynb
%%
\documentclass[11pt,english]{article}

%% This is the automatic preamble used by IPython.  Note that it does *not*
%% include a documentclass declaration, that is added at runtime to the overall
%% document.

\usepackage{amsmath}
\usepackage{amssymb}
\usepackage{graphicx}
\usepackage{ucs}
\usepackage[utf8x]{inputenc}

% needed for markdown enumerations to work
\usepackage{enumerate}

% Slightly bigger margins than the latex defaults
\usepackage{geometry}
\geometry{verbose,tmargin=3cm,bmargin=3cm,lmargin=2.5cm,rmargin=2.5cm}

% Define a few colors for use in code, links and cell shading
\usepackage{color}
\definecolor{orange}{cmyk}{0,0.4,0.8,0.2}
\definecolor{darkorange}{rgb}{.71,0.21,0.01}
\definecolor{darkgreen}{rgb}{.12,.54,.11}
\definecolor{myteal}{rgb}{.26, .44, .56}
\definecolor{gray}{gray}{0.45}
\definecolor{lightgray}{gray}{.95}
\definecolor{mediumgray}{gray}{.8}
\definecolor{inputbackground}{rgb}{.95, .95, .85}
\definecolor{outputbackground}{rgb}{.95, .95, .95}
\definecolor{traceback}{rgb}{1, .95, .95}

% Framed environments for code cells (inputs, outputs, errors, ...).  The
% various uses of \unskip (or not) at the end were fine-tuned by hand, so don't
% randomly change them unless you're sure of the effect it will have.
\usepackage{framed}

% remove extraneous vertical space in boxes
\setlength\fboxsep{0pt}

% codecell is the whole input+output set of blocks that a Code cell can
% generate.

% TODO: unfortunately, it seems that using a framed codecell environment breaks
% the ability of the frames inside of it to be broken across pages.  This
% causes at least the problem of having lots of empty space at the bottom of
% pages as new frames are moved to the next page, and if a single frame is too
% long to fit on a page, will completely stop latex from compiling the
% document.  So unless we figure out a solution to this, we'll have to instead
% leave the codecell env. as empty.  I'm keeping the original codecell
% definition here (a thin vertical bar) for reference, in case we find a
% solution to the page break issue.

%% \newenvironment{codecell}{%
%%     \def\FrameCommand{\color{mediumgray} \vrule width 1pt \hspace{5pt}}%
%%    \MakeFramed{\vspace{-0.5em}}}
%%  {\unskip\endMakeFramed}

% For now, make this a no-op...
\newenvironment{codecell}{}

 \newenvironment{codeinput}{%
   \def\FrameCommand{\colorbox{inputbackground}}%
   \MakeFramed{\advance\hsize-\width \FrameRestore}}
 {\unskip\endMakeFramed}

\newenvironment{codeoutput}{%
   \def\FrameCommand{\colorbox{outputbackground}}%
   \vspace{-1.4em}
   \MakeFramed{\advance\hsize-\width \FrameRestore}}
 {\unskip\medskip\endMakeFramed}

\newenvironment{traceback}{%
   \def\FrameCommand{\colorbox{traceback}}%
   \MakeFramed{\advance\hsize-\width \FrameRestore}}
 {\endMakeFramed}

% Use and configure listings package for nicely formatted code
\usepackage{listingsutf8}
\lstset{
  language=python,
  inputencoding=utf8x,
  extendedchars=\true,
  aboveskip=\smallskipamount,
  belowskip=\smallskipamount,
  xleftmargin=2mm,
  breaklines=true,
  basicstyle=\small \ttfamily,
  showstringspaces=false,
  keywordstyle=\color{blue}\bfseries,
  commentstyle=\color{myteal},
  stringstyle=\color{darkgreen},
  identifierstyle=\color{darkorange},
  columns=fullflexible,  % tighter character kerning, like verb
}

% The hyperref package gives us a pdf with properly built
% internal navigation ('pdf bookmarks' for the table of contents,
% internal cross-reference links, web links for URLs, etc.)
\usepackage{hyperref}
\hypersetup{
  breaklinks=true,  % so long urls are correctly broken across lines
  colorlinks=true,
  urlcolor=blue,
  linkcolor=darkorange,
  citecolor=darkgreen,
  }

% hardcode size of all verbatim environments to be a bit smaller
\makeatletter 
\g@addto@macro\@verbatim\small\topsep=0.5em\partopsep=0pt
\makeatother 

% Prevent overflowing lines due to urls and other hard-to-break entities.
\sloppy

\begin{document}

\section{Some gun violence analysis with Wikipedia data}
As
\href{https://twitter.com/jonst0kes/status/282330530412888064}{requested
by John Stokes}, here are per-capita numbers for gun-related homicides,
relating to GDP and total homicides, so the situation in the United
States can be put in context relative to other nations.

main data source is UNODC (via Wikipedia
\href{http://en.wikipedia.org/wiki/List\_of\_countries\_by\_intentional\_homicide\_rate}{here}
and
\href{http://en.wikipedia.org/wiki/List\_of\_countries\_by\_firearm-related\_death\_rate}{here}).

GDP data from World Bank, again
\href{http://en.wikipedia.org/wiki/List\_of\_countries\_by\_GDP\_(PPP)\_per\_capita}{via
Wikipedia}.

If the numbers on Wikipedia are inaccurate, or their relationship is not
sound (e.g.~numbers taken from different years, during which significant
change occured) then obviously None of this analysis is valid.

To summarize the data, every possible way you look at it the US is lousy
at preventing gun violence. Even when compared to significantly more
violent places, gun violence in the US is a serious problem, and when
compared to similarly wealthy places, the US is an outstanding disaster.

\textbf{UPDATE:} the relationship of the gun data and totals does not
seem to be valid.
\href{http://www2.fbi.gov/ucr/cius2009/offenses/violent\_crime/index.html}{FBI
data} suggests that the relative contribution of guns to homicides in
the US is 47\%, but relating these two data sources gives 80\%. Internal
comparisons should still be fine, but `fraction' analysis has been
stricken.

\begin{codecell}
\begin{codeinput}
\begin{lstlisting}
%load_ext retina
%pylab inline
\end{lstlisting}
\end{codeinput}
\begin{codeoutput}
\begin{verbatim}
Welcome to pylab, a matplotlib-based Python environment [backend: module://IPython.zmq.pylab.backend_inline].
For more information, type 'help(pylab)'.
\end{verbatim}
\end{codeoutput}
\end{codecell}
\begin{codecell}
\begin{codeinput}
\begin{lstlisting}
from IPython.display import display
import pandas
pandas.set_option('display.notebook_repr_html', True)
pandas.set_option('display.precision', 2)
\end{lstlisting}
\end{codeinput}
\end{codecell}
Some utility functions for display

\begin{codecell}
\begin{codeinput}
\begin{lstlisting}
def plot_percent(df, limit=10):
    df['Gun Percent'][:limit].plot()
    plt.ylim(0,100)
    plt.title("% Gun Homicide")
    plt.show()

\end{lstlisting}
\end{codeinput}
\end{codecell}
\begin{codecell}
\begin{codeinput}
\begin{lstlisting}
def plot_percapita(df, limit=10):
    df = df.ix[:,['Homicides', 'Gun Homicides']][:limit]
    df['Total Homicides'] = df['Homicides'] - df['Gun Homicides']
    del df['Homicides']
    df.plot(kind='bar', stacked=True, sort_columns=True)
    plt.ylabel("per 100k")
    plt.show()

\end{lstlisting}
\end{codeinput}
\end{codecell}
\begin{codecell}
\begin{codeinput}
\begin{lstlisting}
def display_relevant(df, limit=10):
    display(df.ix[:,['Homicides', 'Gun Homicides', 'Gun Data Source']][:limit])
\end{lstlisting}
\end{codeinput}
\end{codecell}
Load the data

\begin{codecell}
\begin{codeinput}
\begin{lstlisting}
totals = pandas.read_csv('totals.csv', '\t', index_col=0)
guns = pandas.read_csv('guns.csv', '\t', index_col=0)
gdp = pandas.read_csv('gdp.csv', '\t', index_col=1)
data = totals.join(guns).join(gdp)
data['Gun Percent'] = 100 * data['Gun Homicides'] / data['Homicides']
del data['Unintentional'],data['Undetermined'],data['Gun Suicides']
data = data.dropna()
\end{lstlisting}
\end{codeinput}
\end{codecell}
Of all sampled countries (Found data for 68 countries), the US is in the
top 15 in Gun Homicides per capita.

Numbers are per 100k.

\begin{codecell}
\begin{codeinput}
\begin{lstlisting}
data = data.sort("Gun Homicides", ascending=False)
display_relevant(data, 15)
\end{lstlisting}
\end{codeinput}
\begin{codeoutput}
\begin{verbatim}
Homicides  Gun Homicides Gun Data Source
Country                                                
El Salvador         69.2           50.4     OAS 2011[1]
Jamaica             52.2           47.4     OAS 2011[1]
Honduras            91.6           46.7     OAS 2011[1]
Guatemala           38.5           38.5     OAS 2011[1]
Colombia            33.4           27.1  UNODC 2011 [2]
Brazil              21.0           18.1   UNODC 2011[3]
Panama              21.6           12.9     OAS 2011[1]
Mexico              16.9           10.0   UNODC 2011[4]
Paraguay            11.5            7.3  UNODC 2000[11]
Nicaragua           13.6            7.1     OAS 2011[1]
United States        4.2            3.7  OAS 2012[5][6]
Costa Rica          10.0            3.3   UNODC 2002[7]
Uruguay              5.9            3.2   UNODC 2002[7]
Argentina            3.4            3.0  UNODC 2011[12]
Barbados            11.3            3.0  UNODC 2000[11]
\end{verbatim}
\end{codeoutput}
\end{codecell}
Take top 30 Countries by GDP

\begin{codecell}
\begin{codeinput}
\begin{lstlisting}
top = data.sort('GDP')[-30:]
\end{lstlisting}
\end{codeinput}
\end{codecell}
and rank them by Gun Homicides per capita:

\begin{codecell}
\begin{codeinput}
\begin{lstlisting}
top_by_guns = top.sort("Gun Homicides", ascending=False)
display_relevant(top_by_guns, 5)
plot_percapita(top_by_guns, 10)
\end{lstlisting}
\end{codeinput}
\begin{codeoutput}
\begin{verbatim}
Homicides  Gun Homicides Gun Data Source
Country                                                
United States        4.2            3.7  OAS 2012[5][6]
Israel               2.1            0.9    WHO 2012[10]
Canada               1.6            0.8   Krug 1998[13]
Luxembourg           2.5            0.6    WHO 2012[10]
Greece               1.5            0.6   Krug 1998[13]
\end{verbatim}
\begin{center}
\includegraphics[width=0.7\textwidth]{Gun_Data_orig_files/Gun_Data_orig_fig_00.png}
\par
\end{center}
\end{codeoutput}
\end{codecell}
\textbf{NOTE:} these bar graphs should not be interpreted as fractions
of a total, as the two data sources do not appear to be comparable. But
the red and blue bar graphs should still be internally comparable.

The US is easily \#1 of 30 wealthiest countries in Gun Homicides per
capita, by a factor of 4:1

Adding USA, Canada, and Mexico to all of Europe, USA is a strong \#2
behind Mexico in total gun homicides per-capita

\begin{codecell}
\begin{codeinput}
\begin{lstlisting}
index = (data['Region'] == 'Europe') + \
        (data.index == 'United States') + \
        (data.index == 'Canada') + \
        (data.index == 'Mexico')
selected = data[index]

print "By Total Gun Homicides"
sys.stdout.flush()

by_guns = selected.sort("Gun Homicides", ascending=False)
#by_guns['Gun Homicides'].plot(kind='bar')
plot_percapita(by_guns, limit=25)
display_relevant(selected, limit=None)

\end{lstlisting}
\end{codeinput}
\begin{codeoutput}
\begin{verbatim}
By Total Gun Homicides
\end{verbatim}
\begin{center}
\includegraphics[width=0.7\textwidth]{Gun_Data_orig_files/Gun_Data_orig_fig_01.png}
\par
\end{center}
\begin{verbatim}
Homicides  Gun Homicides Gun Data Source
Country                                                 
Mexico               16.9           10.0   UNODC 2011[4]
United States         4.2            3.7  OAS 2012[5][6]
Montenegro            3.5            2.1    WHO 2012[10]
Moldova               7.5            1.0    WHO 2012[10]
Canada                1.6            0.8   Krug 1998[13]
Serbia                1.2            0.6    WHO 2012[10]
Luxembourg            2.5            0.6    WHO 2012[10]
Greece                1.5            0.6   Krug 1998[13]
Croatia               1.4            0.6    WHO 2012[10]
Switzerland           0.7            0.5     OAS 2011[1]
Malta                 1.0            0.5    WHO 2012[10]
Portugal              1.2            0.5    WHO 2012[10]
Belarus               4.9            0.4   UNODC 2002[7]
Ireland               1.2            0.4    WHO 2012[10]
Italy                 0.9            0.4    WHO 2012[10]
Ukraine               5.2            0.3  UNODC 2000[11]
Estonia               5.2            0.3    WHO 2012[10]
Belgium               1.7            0.3    WHO 2012[10]
Finland               2.2            0.3    WHO 2012[10]
Lithuania             6.6            0.2    WHO 2012[10]
Bulgaria              2.0            0.2    WHO 2012[10]
Georgia               4.3            0.2    WHO 2012[10]
Denmark               0.9            0.2    WHO 2012[10]
France                1.1            0.2    WHO 2012[10]
Netherlands           1.1            0.2    WHO 2012[10]
Sweden                1.0            0.2    WHO 2012[10]
Slovakia              1.5            0.2    WHO 2012[10]
Austria               0.6            0.2    WHO 2012[10]
Latvia                3.1            0.2    WHO 2012[10]
Spain                 0.8            0.1    WHO 2012[10]
Hungary               1.3            0.1    WHO 2012[10]
Czech Republic        1.7            0.1    WHO 2012[10]
Germany               0.8            0.1    WHO 2012[10]
Slovenia              0.7            0.1    WHO 2012[10]
Romania               2.0            0.0    WHO 2012[10]
United Kingdom        1.2            0.0    WHO2012 [10]
Norway                0.6            0.0    WHO 2012[10]
Poland                1.1            0.0    WHO 2012[10]
\end{verbatim}
\end{codeoutput}
\end{codecell}
Let's just compare US, Canada, and UK:

\begin{codecell}
\begin{codeinput}
\begin{lstlisting}
select = data.ix[['United States', 'Canada', 'United Kingdom']]
plot_percapita(select)
\end{lstlisting}
\end{codeinput}
\begin{codeoutput}
\begin{center}
\includegraphics[width=0.7\textwidth]{Gun_Data_orig_files/Gun_Data_orig_fig_02.png}
\par
\end{center}
\end{codeoutput}
\end{codecell}
Normalize to the US numbers (inverse)

\begin{codecell}
\begin{codeinput}
\begin{lstlisting}
select['Homicides'] = select['Homicides']['United States'] / select['Homicides']
select['Gun Homicides'] = select['Gun Homicides']['United States'] / select['Gun Homicides']
display_relevant(select)
\end{lstlisting}
\end{codeinput}
\begin{codeoutput}
\begin{verbatim}
Homicides  Gun Homicides Gun Data Source
United States         1.0            1.0  OAS 2012[5][6]
Canada                2.6            4.9   Krug 1998[13]
United Kingdom        3.5           92.5    WHO2012 [10]
\end{verbatim}
\end{codeoutput}
\end{codecell}
So, you are 2.6 times more likely to be killed in the US than Canada,
and 3.5 times more likely than in the UK. That's bad, but not extreme.

However, you are 4.9 times more likely to be killed \emph{with a gun} in
the US than Canada, and almost 100 times more likely than in the UK.
That is pretty extreme.

Countries represented:

\begin{codecell}
\begin{codeinput}
\begin{lstlisting}
for country in data.index:
    print country
\end{lstlisting}
\end{codeinput}
\begin{codeoutput}
\begin{verbatim}
El Salvador
Jamaica
Honduras
Guatemala
Colombia
Brazil
Panama
Mexico
Paraguay
Nicaragua
United States
Costa Rica
Uruguay
Argentina
Barbados
Montenegro
Peru
Moldova
Israel
India
Canada
Serbia
Luxembourg
Greece
Uzbekistan
Croatia
Kyrgyzstan
Switzerland
Malta
Portugal
Belarus
Ireland
Italy
Kuwait
Ukraine
Estonia
Belgium
Finland
Lithuania
Cyprus
Bulgaria
Georgia
Denmark
France
Netherlands
Sweden
Slovakia
Qatar
Austria
Latvia
New Zealand
Spain
Hungary
Czech Republic
Hong Kong
Australia
Singapore
Chile
Germany
Slovenia
Romania
Azerbaijan
South Korea
United Kingdom
Norway
Japan
Poland
Mauritius
\end{verbatim}
\end{codeoutput}
\end{codecell}
\end{document}
